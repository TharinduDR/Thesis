\newpage
\phantomsection \label{abstract}
\addcontentsline{toc}{chapter}{Abstract}

\chapter*{Abstract}
Semantic Textual Similarity (STS) measures the equivalence of meanings between two textual segments. It is a fundamental task for many natural language processing applications. In this study, we focus on employing STS in the context of translation technology. We start by developing models to estimate STS. We propose a new unsupervised vector aggregation-based STS method which relies on contextual word embeddings. We also propose a novel \textit{Siamese neural network} based on efficient recurrent neural network units. We empirically evaluate various unsupervised and supervised STS methods, including these newly proposed methods in three different English STS datasets, two non-English datasets and a bio-medical STS dataset to list the best supervised and unsupervised STS methods.

We then embed these STS methods in translation technology applications. Firstly we experiment with Translation Memory (TM) systems. We propose a novel TM matching and retrieval method based on STS methods that outperform current TM systems. We then utilise the developed STS architectures in translation Quality Estimation (QE). We show that the proposed methods are simple but outperform complex QE architectures and improve the state-of-the-art results. The implementations of these methods have been released as open source.



