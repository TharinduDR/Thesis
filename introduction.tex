\addcontentsline{toc}{chapter}{Introduction}

\chapter*{Introduction}
\label{cha:introduction}


Semantic textual similarity (STS) is a natural language processing (NLP) task to quantitatively assess the semantic similarity between two text snippets. STS is a fundamental NLP task for many text-related applications, including text de-duplication, paraphrase detection, semantic searching, and question answering. Measuring STS is a machine learning (ML) problem, where a ML model predicts a value that represents the similarity of the two input texts. These machine learning models can be categorised into two main areas; supervised and unsupervised. Supervised STS ML models have been trained on an annotated STS dataset while the unsupervised STS ML models predict STS without being trained on annotated STS data. In \textbf{Part I} of the thesis, we explore supervised and unsupervised ML models in STS. We explore embedding aggregation based STS methods, sentence encoders, Siamese neural networks and transformers in STS. Furthermore, for each STS method, we analyse the ability of the model to perform in a multilingual and multi-domain setting. In the process, we introduce a new state-of-the-art unsupervised vector aggregation based STS method developed on contextual word embeddings and a new state-of-the-art supervised STS method based on Siamese neural networks and classical word embedding models. 

The second and third parts of the thesis, focus on applying the developed STS methods in the applications of translation technology; translation memories (TM) and translation quality estimation (QE). In \textbf{Part II} of the thesis, we identify that the edit distance based matching and retrieval algorithms in TMs are unable to capture the similarity between segments and as a result even if the TM contains a semantically similar segment, the retrieval algorithm will not be able to identify it. To overcome this, we propose semantically powerful algorithms for TM matching and retrieval. Considering the efficiency, we employ a TM matching and retrieval algorithm based on sentence encoders we experimented in Part I of the thesis. We empirically show that this algorithm outperforms edit distance-based matching algorithms paving a new direction for TMs.

As the next application, we utilise the STS architectures we developed, in translation quality estimation. We identify that current state-of-the-art neural QE models are very complex and require a lot of computing resources. To overcome this, we remodel the QE task as a crosslingual STS task.  We show that STS architectures can be successfully applied in QE by changing the input embeddings into crosslingual embeddings and they are very simple and efficient compared to the current state-of-the-art QE models. Based on that, we develop TransQuest - a new state-of-the-art QE framework that won the WMT 2020 QE shared task. TransQuest supports both word-level and sentence-level QE and has been evaluated on more than 15 language pairs. Furthermore, for the first time, we explore multilingual QE models with TransQuest, focussing on low resource languages. We release TransQuest as an open-source QE framework and by the time of writing this, TransQuest has more than 9,000 downloads from the community. 

The research questions in this work can be summarised as the following:

\textbf{RQ1} What are the available supervised and unsupervised STS methods and how do they perform in multilingual and multi-domain environments? 

\textbf{RQ2} Can the neural STS methods be applied in TMs? How efficient and effective are they compared to the real-world TM tools?

\textbf{RQ3} Can the state-of-the-art STS methods be adopted in QE task? Can these simple STS architectures outperform current complex QE methods? 

Each part of the thesis will address these research questions separately.

