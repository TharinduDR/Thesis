\chapter{\label{cha:sts_introduction}Introduction}


\section{What is Semantic Textual Similarity?}
\cite{cer-etal-2017-semeval} 
\section{Related Work}

\section{Datasets}
We experimented with several datasets throughout the experiments in the Semantic Textual Similarity Section. In order to maintain the versatility of our methods we experimented with several English datasets as well as several non English datasets and several datasets from different domains which we will introduce in this section.

\subsection{English Datasets}
\begin{enumerate}
  \item \textbf{SICK dataset} - The SICK data contains 9927 sentence pairs with a 5,000/4,927 training/test split which were employed in the SemEval 2014 Task1: Evaluation of Compositional Distributional Semantic Models on Full Sentences through Semantic Relatedness and Textual Entailment \cite{marelli-etal-2014-semeval}. The dataset has two types of annotations: Semantic Relatedness and Textual Entailment. We only use Semantic Relatedness annotations in our research. SICK was built starting from two existing datasets: the 8K ImageFlickr data set \footnote{The 8K ImageFlickr data set is available at \url{http://hockenmaier.cs.illinois.edu/8k-pictures.html}} \cite{rashtchian-etal-2010-collecting} and the SemEval-2012 STS MSR-Video Descriptions dataset \footnote{The SemEval-2012 STS MSR-Video Descriptions dataset is available at \url{https://www.cs.york.ac.uk/semeval-2012/task6/index.html}} \cite{agirre-etal-2012-semeval}. The 8K ImageFlickr dataset is a dataset of images, where each image is associated with five descriptions. To derive SICK sentence pairs the organisers randomly selected 750 images and sampled two descriptions from each of them. The SemEval2012 STS MSR-Video Descriptions data set is a collection of sentence pairs sampled from the short video snippets which compose the Microsoft Research Video Description Corpus \footnote{The Microsoft Research Video Description Corpus is available to download at \url{https://research.microsoft.com/en-us/downloads/38cf15fd-b8df-477e-a4e4-a4680caa75af/}}. A subset of 750 sentence pairs were randomly chosen from this data set to be used in SICK. 
  
  In order to generate SICK data from the 1,500 sentence pairs taken from the source data sets, a 3-step process was applied to each sentence composing the pair, namely (i) normalization, (ii) expansion and (iii) pairing.
  
  \item \textbf{STS 2017 Dataset}
  
\end{enumerate}

\section{Applications}

%%% The following are used by emacs, and similar:

%%% Local Variables: ***
%%% TeX-master: "../thesis.tex"  ***
%%% End: ***
